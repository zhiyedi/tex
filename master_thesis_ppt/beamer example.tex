%声明文档类型和比例
\documentclass[aspectratio=169, 10pt, utf8, mathserif]{beamer}
%调用相关的宏包
\usepackage{ctex}
\usepackage{amsmath, amsfonts}
\usepackage{graphicx}
\usepackage{multicol} %分栏
\usepackage{booktabs} %表格功能包
\usepackage{multirow} %合并多行表格
\usepackage{enumerate} %有序编号
\usepackage{listings} %代码包
\usepackage{xcolor} %代码高亮包
\lstset{
	language=Matlab, %代码语言使用的是matlab
	frame=shadowbox, %把代码用带有阴影的框圈起来
	rulesepcolor=\color{red!20!green!20!blue!20}, %代码块边框为淡青色
	keywordstyle=\color{blue}\bfseries, %代码关键字的颜色为蓝色,粗体
	commentstyle=\color{red}\textit, %设置代码注释的颜色
	showstringspaces=false, %不显示代码字符串中间的空格标记
	numbers=left, %显示行号
	numberstyle=\tiny, %行号字体
	stringstyle=\ttfamily, %代码字符串的特殊格式
	breaklines=true, %过长的代码自动换行
	extendedchars=false,  %解决代码跨页时,章节标题,页眉等汉字不显示的问题
	escapebegin=\begin{CJK*}{GBK}{hei},escapeend=\end{CJK*} %防止中文报错
	texcl=true}

\usetheme{Rochester} %主题包之一,直接换名字即可
\usecolortheme{default} %主题色之一,直接换名字即可。
\usefonttheme{professionalfonts}

% 设置用acrobat打开就会全屏显示
\hypersetup{pdfpagemode=FullScreen}

% 设置logo
\pgfdeclareimage[height=1.5cm, width=1.5cm]{university-logo}{120701101}
\logo{\pgfuseimage{university-logo}}
\setbeamertemplate{bibliography item}[text]

%--------------正文开始---------------
\begin{document}

%每个章节都有小目录
\AtBeginSection[]
{
 \begin{frame}<beamer>
   \tableofcontents[currentsection]
 \end{frame}
}

\title{Hilbert第五问题和局部紧群的相关研究}

\author{李志刚 \\ \quad \\ 指导教师:邓少强 教授}
\institute
{
	南开大学 \\
	数学科学学院
}
\date{\today}
\begin{frame}
    %\maketitle
    \titlepage
\end{frame}

\begin{frame}
	\frametitle{目录}
	\tableofcontents[hideallsubsections]
\end{frame}

\section{背景介绍}

\begin{frame}[plain]
	\frametitle{背景介绍}
	\begin{Definition}
		一个拓扑群$G$是一个集合,这个集合首先是一个拓扑空间,同时也是一个群,并且群乘法$(g_1,g_2)\to g_1g_2(\forall g_1,g_2\in G)$是一个$G \times G$到$G$的连续映射,同时逆运算$g\to g^{-1}(\forall g\in G)$是$G$到$G$的连续映射.
	\end{Definition}

	如果拓扑群$G$还是一个光滑流形(不要求流形是第二可数的),群乘法和逆运算都是光滑的,则$G$是一个李群,如果额外要求流形是解析的,并且群乘法和逆运算都是解析的,则称$G$是一个解析李群.显然李群是拓扑群,反之,什么样的拓扑群是李群呢?
\end{frame}

\begin{frame}[plain]
	\frametitle{背景介绍}

	\begin{Theorem}[Hilbert第五问题]
		局部欧的拓扑群是李群.
	\end{Theorem}
	1952年Gleason证明了有限维没有小子群(no small subgroups)的局部紧群是李群.同年Montgomery和Zippin的结果蕴含着局部欧的拓扑群没有小子群,两者结合便得到了Hilbert第五问题的答案. 
	
	2010年, Goldbring用非标准分析的方法也得到了一个新的证明.
	\begin{Theorem}[Gleason-Yamabe定理]
		$G$是一个局部紧群,则对于$G$的每个单位连通邻域$U$,都存在一个$G$的子群$G'$和一个$G'$的紧正规子群$K$,其中$G'$是开子群, $K\subset U$,则$G'/K$同构于一个李群.
	\end{Theorem}
	1953年, Yamabe将Gleason的结果中的有限维条件去掉了,并且得到了Gleason-Yamabe定理.
\end{frame}

\begin{frame}[plain]
	\frametitle{背景介绍}
	\begin{example}
		任何群取离散拓扑,都是$0$维李群.有理数加法群$\mathbb{Q}$取实直线诱导的子拓扑是一个拓扑群,其连通分支是单点集,并且其任意一个开集都不能同胚于一个欧式空间的开集,所以不是李群.
	\end{example}
	\begin{example}
		$T^n=S^1\times S^1\cdots \times S^1$是一个李群,更是拓扑群.无穷多个$S^1$的直积不是一个李群, 但是它是一个拓扑群.
	\end{example}
	\begin{example}
		$p$是一个素数,在$\mathbb{Z}$上定义一个p-adic范数$\Vert\cdot\Vert$, $\Vert n\Vert=\frac{1}{p^i}$,其中对于非负正数$i$, $p^i$能整除$n$,但是$p^{i+1}$不能整除$n$.用这个范数产生$\mathbb{Z}$上的度量,将这个度量空间完备化,得到p-adic整数群$\mathbb{Z}_p$,它同胚于Cantor集,自然不是一个李群,但是$\mathbb{Z}_p$是一个拓扑群.
	\end{example}
\end{frame}

\begin{frame}[plain]
	\frametitle{背景介绍}

	\begin{block}{Hilbert-Smith猜想v1}
		如果局部紧群$G$能连续且忠实地作用在一个连通流形上,则$G$同构一个李群.
	\end{block}
	这个猜想还有几个等价的版本,这里再给出一个.
	\begin{block}{Hilbert-Smith猜想v2}
		任意连通流形上都不存在$p-adic$整数$\mathbb{Z}_p$的忠实作用.
	\end{block}

	2013年,Pardon解决了这个连通流形是三维的Hilbert-Smith猜想.
\end{frame}

\section{基本概念}

\begin{frame}[plain]
	\frametitle{基本概念}
	\begin{Definition}
		局部拓扑群,或者简称为局部群:$G$是一个拓扑空间, $\Omega\subset G\times G,\Lambda\subset G$, $e\in G$, $*$是一个从$\Omega$到$G$的连续映射(称*为$G$中的乘法), $\tau$是一个从$\Lambda$到$G$的连续映射(称$\tau$为$G$中的取逆运算),称$(G,\Omega,\Lambda,e,*,\tau)$或$G$为局部群是指满足下面这些性质
		\begin{enumerate}
			\item[\textnormal{(1)}]$\Omega$和$\Lambda$分别是$G\times G$和$G$中的开集,并且$\left\{e\right\}\times G \subset \Omega,G\times \left\{e\right\} \subset \Omega,e\in \Lambda$.
			\item[\textnormal{(2)}]如果$a,b,c\in G$,且$a*(b*c)$和$(a*b)*c$这两项都有定义,则$a*(b*c)=(a*b)*c$.(允许其中一项有定义,而另一项没有定义)
			\item[\textnormal{(3)}]$\forall g\in G,e*g=g*e=g$.
			\item[\textnormal{(4)}]$\forall h\in \Lambda,h*\tau(h)=\tau(h)*h=e$.
		\end{enumerate}
	\end{Definition}
	显然一个拓扑群自然是一个局部群,将一个拓扑群$G$的乘法运算和取逆运算限制到一个单位开邻域$V$上,则$V$是一个局部群.如果两个局部群在一个充分小的单位邻域上是一样的,则认为这两个局部群是同构的.如果$G$是一个光滑流形, $*$和$\tau$都是光滑的,那么$G$称为一个局部李群.
\end{frame}

\begin{frame}[plain]
	\frametitle{基本概念}
	\begin{Definition}
		假设$\varphi:G\to H$是两个局部群$G,H$之间的连续映射, $\varphi$是局部群同态是指$\varphi$满足下面这些性质
		\begin{enumerate}
			\item[\textnormal{(1)}]保持单位元$\varphi(e_G)=e_H$.
			\item[\textnormal{(2)}]保持逆运算$\varphi(g^{-1})=(\varphi(g))^{-1}$,因为局部群不是封闭的,所以还要要求$g^{-1},(\varphi(g))^{-1}$都有定义.
			\item[\textnormal{(3)}]如果$g,h\in G$并且$gh$有定义, $\varphi(g)\varphi(h)$也有定义,则$\varphi(g)\varphi(h)=\varphi(gh)$.
		\end{enumerate}
	\end{Definition}

	类似的还可以把李群上的左不变向量场和李代数都推广到局部上.
\end{frame}

\begin{frame}[plain]
	\frametitle{基本概念}
	\begin{Theorem}
		如果在局部欧的拓扑群$G$的某个单位开邻域$U$内, $V$是对称邻域,并且$V^2\subset U$,将拓扑群的乘法和取逆运算限制到$V$上,使得$V$成为一个局部群.如果$V$是一个解析的局部李群,那么$G$有解析李群结构.
	\end{Theorem}
	如果拓扑群是一个局部(local)李群,那么它也是一个整体(global)李群.
\end{frame}

\section{Gleason-Yamabe定理}
\section{Hilbert第五问题}
\begin{frame}
	\zihao{-2}\centering{\textbf{谢谢各位专家的聆听!}}
\end{frame}
\end{document}