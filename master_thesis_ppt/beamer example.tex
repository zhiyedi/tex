%声明文档类型和比例
\documentclass[aspectratio=169, 10pt, utf8, mathserif]{beamer}
%调用相关的宏包
\usepackage{ctex}
\usepackage{amsmath, amsfonts}
\usepackage{graphicx}
\usepackage{multicol} %分栏
\usepackage{booktabs} %表格功能包
\usepackage{multirow} %合并多行表格
\usepackage{enumerate} %有序编号
\usepackage{listings} %代码包
\usepackage{xcolor} %代码高亮包
\lstset{
	language=Matlab, %代码语言使用的是matlab
	frame=shadowbox, %把代码用带有阴影的框圈起来
	rulesepcolor=\color{red!20!green!20!blue!20}, %代码块边框为淡青色
	keywordstyle=\color{blue}\bfseries, %代码关键字的颜色为蓝色,粗体
	commentstyle=\color{red}\textit, %设置代码注释的颜色
	showstringspaces=false, %不显示代码字符串中间的空格标记
	numbers=left, %显示行号
	numberstyle=\tiny, %行号字体
	stringstyle=\ttfamily, %代码字符串的特殊格式
	breaklines=true, %过长的代码自动换行
	extendedchars=false,  %解决代码跨页时,章节标题,页眉等汉字不显示的问题
	escapebegin=\begin{CJK*}{GBK}{hei},escapeend=\end{CJK*} %防止中文报错
	texcl=true}

\usetheme{Rochester} %主题包之一,直接换名字即可
\usecolortheme{default} %主题色之一,直接换名字即可。
\usefonttheme{professionalfonts}

% 设置用acrobat打开就会全屏显示
\hypersetup{pdfpagemode=FullScreen}

% 设置logo
\pgfdeclareimage[height=1.5cm, width=1.5cm]{university-logo}{120701101}
\logo{\pgfuseimage{university-logo}}
\setbeamertemplate{bibliography item}[text]

%--------------正文开始---------------
\begin{document}

%每个章节都有小目录
\AtBeginSection[]
{
 \begin{frame}<beamer>
   \tableofcontents[currentsection]
 \end{frame}
}

\title{Hilbert第五问题和局部紧群的相关研究}

\author{李志刚 \\ \quad \\ 指导教师:邓少强 教授}
\institute
{
	南开大学 \\
	数学科学学院
}
\date{\today}
\begin{frame}
    %\maketitle
    \titlepage
\end{frame}

\begin{frame}
	\frametitle{目录}
	\tableofcontents[hideallsubsections]
\end{frame}

\section{背景介绍}

\begin{frame}[plain]
	\begin{Theorem}
		(Hilbert第五问题)局部欧的拓扑群是李群.
	\end{Theorem}
	这个问题最终在1952年被Deane Montgomery, Leo Zippin和Andrew M. Gleason解决. Gleason证明了有限维没有小子群(no small subgroups)的局部紧群是李群,而Montgomery和Zippin的结果蕴含着局部欧的拓扑群没有小子群,两者结合便得到了Hilbert第五问题的答案. 2010年, Goldbring用非标准分析的方法也得到了一个新的证明.
\end{frame}

\begin{frame}[plain]
	\frametitle{Hilbert第五问题}
	\begin{Definition}
		一个拓扑群$G$是一个集合,这个集合首先是一个拓扑空间,同时也是一个群,并且群乘法$(g_1,g_2)\to g_1g_2(\forall g_1,g_2\in G)$是一个$G \times G$到$G$的连续映射,同时逆运算$g\to g^{-1}(\forall g\in G)$是$G$到$G$的连续映射.
	\end{Definition}

	如果拓扑群$G$还是一个光滑流形(不要求流形是第二可数的),群乘法和逆运算都是光滑的,则$G$是一个李群,如果额外要求流形是解析的,并且群乘法和逆运算都是解析的,则称$G$是一个解析李群.

	\begin{Theorem}
		(Hilbert第五问题)局部欧的拓扑群是李群.
	\end{Theorem}
	
\end{frame}

\begin{frame}[plain]
	\frametitle{Hilbert第五问题}
	\begin{example}
		任何群取离散拓扑,都是$0$维李群.有理数加法群$\mathbb{Q}$取实直线诱导的子拓扑是一个拓扑群,其连通分支是单点集,并且其任意一个开集都不能同胚于一个欧式空间的开集,所以不是李群.
	\end{example}
	\begin{example}
		$T^n=S^1\times S^1\cdots \times S^1$是一个李群,更是拓扑群.无穷多个$S^1$的直积不是一个李群, 但是它是一个拓扑群.
	\end{example}
	\begin{example}
		$p$是一个素数,在$\mathbb{Z}$上定义一个p-adic范数$\Vert\cdot\Vert$, $\Vert n\Vert=\frac{1}{p^i}$,其中对于非负正数$i$, $p^i$能整除$n$,但是$p^{i+1}$不能整除$n$.用这个范数产生$\mathbb{Z}$上的度量,将这个度量空间完备化,得到p-adic整数群$\mathbb{Z}_p$,它同胚于Cantor集,自然不是一个李群,但是$\mathbb{Z}_p$是一个拓扑群.
	\end{example}
\end{frame}


\begin{frame}[plain]
	\frametitle{基本概念}

\begin{Theorem}
(i) 设$A$为一个左对称代数,则交换子\[[x,y]=xy-yx, \forall x,y\in A.\]
自然定义了一个李代数$g(A)$,其被称为$A$的邻接李代数,同时$A$被称为李代数$g(A)$上的一个可兼容的左对称代数结构.

(ii) 对于任意左对称代数$A$,自然的可以定义左乘算子与右乘算子,\[L:A\rightarrow gl(A), L(x)(y)=xy;\quad R:A\rightarrow gl(A), R(x)(y)=yx.\]
由此方程\text{(1)}可等价的表示为
\begin{equation}
[L(x),L(y)]=L([x,y]), \forall x,y\in g(A).
\end{equation}
\end{Theorem}
\end{frame}

\begin{frame}
	\frametitle{基本概念}
一个左对称代数可以自然定义一个李代数,那么一个自然的问题是,李代数上是否一定有左对称代数结构.

实际上,半单李代数上是没有左对称代数结构的.寻求三维以下的李代数上的左对称代数结构有较完善的结果方法,但在更高的维度上,这一课题的一般理论还正在发展中.

\end{frame}





\begin{frame}
	\frametitle{基本概念}
	左对称代数$A$的一些子类也被广大学者研究,在各个几何或数学物理等领域都有重要的作用,主要的有以下几个子类.

	\begin{itemize}
	  \item[(1)] 结合代数:定义的结合子 $(x,y,z)$等于0;
	  \item[(2)] 完备左对称代数:对$\forall x\in A$,右乘算子$R(x)$是幂零的
	  
	  \begin{center}
		判定定理:对任意的$x\in A, tr(R(x))=0 \Leftrightarrow$ 左对称代数$A$是完备的;
	  \end{center}
	  \item[(3)] 左对称导子代数与左对称内导子代数:当左对称代数$A$对$\forall x\in A$,有$L(x)$或者$R(x)$为子伴随李代数$g(A)$的一个导子(内导子);
	  \item[(4)] Novikov代数:当左对称代数$A$满足条件
	  $$(xy)z=(xz)y,\forall x,y,z\in A.$$即Novikov代数是右乘算子可交换的左对称李代数;
	  \item[(5)] 双对称代数:当左对称代数$A$在相同的乘积下满足右对称.
	\end{itemize}
\end{frame}

\section{Gleason-Yamabe定理}
\begin{frame}
	\frametitle{基本概念}
	\begin{definition}
	广义Oscillator李代数$\mathfrak{g}_m(\boldmath{\lambda})=\mathfrak{g}_m(\lambda_1,\cdots,\lambda_m)$是以$(P,X_1,\cdots,X_m,Y_1,\cdots,Y_m,Q)$为基底,满足以下非零交换关系的实$(2m+2)$维实可解李代数:
	$$ [X_i,Y_i]=P,\quad [X_i,Q]=-\lambda_iY_i,\quad [Y_i,Q]=\lambda_iX_i .\quad \lambda_i\in\mathbb{R}^+.  $$
    \end{definition}
此类李代数是唯一一类可容许一个双不变Lorentzian度量的非交换可解李代数.几何上,Oscillator李群$G_m(\boldmath{\lambda})$是以$\mathfrak{g}_m(\boldmath{\lambda})$为李代数的连通单连通李群.这类李群是唯一的可容许一个双不变Lorentzian度量的连通单连通非交换的可解李群.

特别的,在同构意义下仅有一个四维的Oscillator李代数,在后文中Oscillator李代数特指四维情形	
$$L=<e_1,e_2,e_3,e_4|[e_1,e_2]=e_3,[e_4,e_1]=e_2,[e_4,e_2]=-e_1>.$$
\end{frame}

\section{Hilbert第五问题}
\subsection{Mohammed Guediri的研究成果}
\begin{frame}
	\frametitle{Mohammed Guediri的研究成果}
	M. Guediri在2014年发展了对给定的低维可解李代数,寻求其上的完备左对称代数结构的方法.应用于Oscillator李代数$L$得到以下两类完备左对称代数:

	$1.A(s,t)$:存在一组实数$s,t$,使得在基底$\{e_1,e_2,e_3,e_4\}$下,非零的左对称关系为:
	\begin{equation*}\left\{
	   \begin{array}{l}
		e_1\cdot e_1=te_3, e_1\cdot e_2=\frac{1}{2}e_3,\\
		e_2\cdot e_1=-\frac{1}{2}e_3, e_2\cdot e_2=te_3,\\
		e_4\cdot e_1=e_2, e_4\cdot e_2=-e_1,e_4\cdot e_4=se_3.
	   \end{array}\right.
	\end{equation*}

	$2.B_4$:在基底$\{e_1,e_2,e_3,e_4|[e_1,e_2]=e_3,[e_4,e_1]=e_2,[e_4,e_2]=-e_1\}$下,非零的左对称关系为:
	\begin{equation*}\left\{
		\begin{array}{l}
		    e_1\cdot e_1=e_4, e_1\cdot e_2=\frac{1}{2}e_3,\\
	  e_2\cdot e_1=-\frac{1}{2}e_3, e_2\cdot e_2=e_4,\\
      e_4\cdot e_1=e_2, e_4\cdot e_2=-e_1.
		\end{array}\right.
	 \end{equation*}
\end{frame}

\subsection{本文的主要结果}
\begin{frame}
	\frametitle{本文的主要结果}
		考虑Oscillator李代数$L=<e_1,e_2,e_3,e_4|[e_1,e_2]=e_3,[e_4,e_1]=e_2,[e_4,e_2]=-e_1>$.用$R_i$表示右乘算子$R_{e_i},L_i$表示左乘算子$L_{e_i},ad_i$表示伴随算子$ad_{e_i}$,其中$1\leq i\leq 4 $.设
	
		$$
		R_{i}=\left(\begin{smallmatrix}
		R_{11}^{i} & R_{12}^{i} & R_{13}^{i} & R_{14}^{i} \\
		R_{21}^{i} & R_{22}^{i} & R_{23}^{i} & R_{24}^{i} \\
		R_{31}^{i} & R_{32}^{i} & R_{33}^{i} & R_{34}^{i} \\
		R_{41}^{i} & R_{42}^{i} & R_{43}^{i} & R_{44}^{i}
		\end{smallmatrix}\right).
		$$
		\noindent 由$L_i-R_i=ad_i$,可以得到$L_i$.

		考虑$e_2e_4$,其即可表示为$e_2$左乘$e_4$,也可表示为$e_4$右乘$e_2$,所以$L_2$的第四列元素与$R_4$的第二列对应元素相等,即有
	$$ R_{14}^2+1=R_{12}^4,\ R_{24}^2=R_{22}^4,\ R_{34}^2=R_{32}^4,\ R_{44}^2=R_{42}^4. $$
	同理可由$e_1e_2$得到 $R_{12}^1=R_{11}^2,\ R_{22}^1=R_{21}^2,\ R_{32}^1+1=R_{31}^2,\ R_{42}^1=R_{41}^2; $
	由$e_1e_4$得到 $R_{14}^1=R_{11}^4,\ R_{24}^1-1=R_{21}^4,\ R_{34}^1=R_{31}^4,\ R_{44}^1=R_{41}^4, $
\end{frame}



\begin{frame}
	\frametitle{本文的主要结果}
	
	$$B=\left(\begin{smallmatrix}
	B_{11} & 0 & 0 & 0 \\
	0 & B_{11} & 0 & 0 \\
	0 & 0 & 0 & B_{11} \\
	0 & 0 & B_{11} & 0
	\end{smallmatrix}\right).
	$$
	不失一般性,取$B_{11}=1$进行计算.
	
	为了计算方便,我们仅考虑$gg=[g, g]=<e_{1}, e_{2}, e_{3}>$的情况.
	
	由$[L_1,L_2]=L_3=0$可得如下方程组:

	\begin{equation*}\left\{
		\begin{array}{l}
		R_{22}^{2} R_{24}^{1}=0, \\
		\left(R_{24}^{1}\right)^{2}+R_{24}^{1}-1=0, \\
		\left(R_{24}^{1}\right)^{2}-3 R_{24}^{1}+1=0, \\
		R_{14}^{1}=R_{34}^{1}=0.
		\end{array}
		\right.
		\end{equation*}
		
		显然该方程组没有解.

\end{frame}
	
\begin{frame}
	\frametitle{本文的主要结果}
	通过观察计算过程,我们发现对称非退化不变双线性型这一条件过于苛刻,因此我们利用此条件得到的结论$R_3=L_3=0$进行计算,具体如下.

基于假设$R_3=L_3=0,$以及$gg\subset [g,g]=\langle e_1,e_2,e_3 \rangle$,可得
$$L_1=\left(\begin{smallmatrix}
R_{11}^{1} & R_{12}^{1} & 0 & R_{14}^{1} \\
R_{21}^{1} & R_{22}^{1} & 0 & R_{24}^{1}-1 \\
R_{31}^{1} & R_{32}^{1}+1 & 0 & R_{34}^{1} \\
0 & 0 & 0 & 0
\end{smallmatrix}\right),
L_2=\left(\begin{smallmatrix}
R_{12}^{1} & R_{12}^{2} & 0 & R_{14}^{2}+1 \\
R_{22}^{1} & R_{22}^{2} & 0 & R_{24}^{2} \\
R_{32}^{1} & R_{32}^{2} & 0 & R_{34}^{2} \\
0 & 0 & 0 & 0
\end{smallmatrix}\right),L_3=0, L_4=\left(\begin{smallmatrix}
R_{14}^{1} & R_{14}^{2} & 0 & R_{14}^{4} \\
R_{24}^{1} & R_{24}^{2} & 0 & R_{24}^{4} \\
R_{34}^{1} & R_{34}^{2} & 0 & R_{34}^{4} \\
0 & 0 & 0 & 0
\end{smallmatrix}\right). $$

接下来考虑左对称结构条件:\[I:[L_1,L_2]=L_3=0;\ II:[L_1,L_4]=-L_2;\ III:[L_2,L_4]=L_1.\]
由此类条件可以得到由17个未知数构成的27个方程\footnote{$R_{34}^4$未在方程组中显现,因此可以取任意实数},我们以$(i,j)_k,i=1,2,3;j=1,2,4;k=I,II,II$来表示由条件$k$中的$(i,j)$元得到的方程,如
\[(1,2)_I:R_{11}^1R_{12}^2+R_{12}^1R_{22}^2-R_{12}^1R_{12}^1-R_{12}^2R_{22}^1=0.\]
\end{frame}

\begin{frame}[plain]
	\frametitle{本文的主要结果}
	观察上述矩阵的性质,通过计算可以解出某些未知数,同时能得到一些关系方程,具体过程如下.

	\noindent 由$(1,1)_{II},(2,2)_{II}$可得
	$$R_{12}^1+R_{22}^2=0.\eqno(1)$$
	由$(1,1)_{III},(2,2)_{III}$可得
	$$R_{11}^1+R_{22}^1=0.\eqno(2)$$
	将方程(1)与(2)代入$(1,2)_I$及$(2,1)_I$可得
	$$R_{11}^1R_{12}^2=(R_{12}^1)^2.\eqno(3)$$
	$$R_{12}^1R_{21}^1=-(R_{11}^1)^2.\eqno(4)$$
	联立方程$(1,4)_I$与$(2,4)_I$可得
	$$2R_{11}^1R_{12}^1(R_{24}^2-R_{14}^1)=0.\eqno(5) $$
	经过对方程(5)的讨论,可得Oscillator李代数上的左对称代数结构由以下方程组决定.
\end{frame}

\begin{frame}
	\frametitle{本文的主要结果}

	\begin{equation*}\left\{
	\begin{array}{l}
	R_{11}^1=R_{12}^1=R_{21}^1=R_{22}^1=R_{14}^1=R_{12}^2=R_{22}^2=R_{24}^2=0,\\
	R_{24}^1=1,\\
	R_{32}^1=-\frac{1}{2},\\
	R_{14}^2=-1,\
	R_{31}^1=R_{32}^2,\\
	R_{24}^4+2R_{31}^1R_{14}^4+2R_{34}^2=0,\\
	-R_{14}^4+2R_{31}^1R_{24}^4-2R_{34}^1=0,\\
	R_{34}^4=s.\quad s\in \mathbb{R}
	\end{array}
	\right. \eqno(6)
	\end{equation*}	

	上述方程组得到右乘矩阵(有五个未知元$(R_{31}^{1},R_{34}^{1},R_{34}^{2},R_{14}^{4},R_{24}^{4})$)
	$$R_1=\left(\begin{smallmatrix}
	0 & 0 & 0 & 0 \\
	0 & 0 & 0 & 1 \\
	R_{31}^{1} & -\frac{1}{2} & 0 & R_{34}^{1} \\
	0 & 0 & 0 & 0
	\end{smallmatrix}\right),
	R_2=\left(\begin{smallmatrix}
	0 & 0 & 0 & -1 \\
	0 & 0 & 0 & 0 \\
	\frac{1}{2} & R_{31}^{1} & 0 & R_{34}^{2} \\
	0 & 0 & 0 & 0
	\end{smallmatrix}\right),R_3=0,R_4=\left(\begin{smallmatrix}
	0 & 0 & 0 & R_{14}^{4} \\
	0 & 0 & 0 & R_{24}^{4} \\
	R_{34}^{1} & R_{34}^{2} & 0 & s \\
	0 & 0 & 0 & 0
	\end{smallmatrix}\right).$$
	据完备左对称代数的判定条件,由$tr(R_i)=0$,可知该类左对称代数结构是完备的.
\end{frame}

\begin{frame}[plain]
	\frametitle{本文的主要结果}
	\begin{Example}
		取$R_{31}^1=R_{34}^1=R_{24}^4=1,\ R_{14}^4=0,\ R_{34}^2=-\frac{1}{2}$,则由上述结果,可取得Oscillator李代数上的一类左对称代数,其满足以下方程组.
		
		\begin{equation*}\left\{
		\begin{array}{l}
		[e_1,e_2]=e_3,[e_4,e_1]=e_2,[e_4,e_2]=-e_1,\\
		e_1e_1=e_3,e_1e_2=\frac{1}{2}e_3,e_1e_4=e_3,\\
		e_2e_1=-\frac{1}{2}e_3,e_2e_2=e_3,e_2e_4=-\frac{1}{2}e_3,\\
		e_4e_1=e_2+e_3,e_4e_2=-e_1-\frac{1}{2}e_3,e_4e_4=e_2+se_3.
		\end{array}
		\right.
		\end{equation*}

该例子显然不包含在Mohammed Guediri给出的两类中.
	
	\end{Example}
\end{frame}

\begin{frame}
	\frametitle{本文的主要结果}
	下面我们尝试得出Oscillator李代数上的非完备左对称代数结构.
	通过观察上述结果,我们发现$R_{44}^4$对计算过程的影响不大,同时该元素若是非零的,则可以使$tr(R_4)\neq 0$,从而得到一类非完备的左对称代数结果,具体计算过程如下.设
	$$L_1=\left(\begin{smallmatrix}
	0 & 0 & 0 & 0 \\
	0 & 0 & 0 & 0 \\
	R_{31}^{1} & \frac{1}{2} & 0 & R_{34}^{1} \\
	R_{41}^{1} & R_{42}^{1} & 0 & R_{44}^{1}
	\end{smallmatrix}\right),
	L_2=\left(\begin{smallmatrix}
	0 & 0 & 0 & 0 \\
	0 & 0 & 0 & 0 \\
	-\frac{1}{2} & R_{31}^{1} & 0 & R_{34}^{2} \\
	R_{42}^{1} & R_{42}^{2} & 0 & R_{44}^{2}
	\end{smallmatrix}\right), L_3=0, L_4=\left(\begin{smallmatrix}
	0 & -1 & 0 & R_{14}^{4} \\
	1 & 0 & 0 & R_{24}^{4} \\
	R_{34}^{1} & R_{34}^{2} & 0 & s \\
	R_{44}^{1} & R_{44}^{2} & 0 & R_{44}^{4}
	\end{smallmatrix}\right).$$
	考虑左对称结构条件
	\[I:[L_1,L_2]=L_3=0;\ II:[L_1,L_4]=-L_2;\ III:[L_2,L_4]=L_1.\]
	以$(i,j)_k,i=1,2,3;j=1,2,4;k=I,II,II$来表示由条件$k$中的$(i,j)$元得到的方程.
\end{frame}

\begin{frame}[plain]
	\frametitle{本文的主要结果}
	假设$R_{14}^4\neq 0,\ R_{24}^4\neq 0$,则由方程$(1,1)_I,(1,2)_I,(1,4)_I$,以及方程$(1,2)_{II},(1,4)_{II}$,可得
	$$R_{41}^1=R_{42}^1=R_{44}^1=R_{42}^2=R_{44}^2=0. $$
	代入剩余方程组可得
\begin{equation*}\left\{
\begin{array}{l}
R_{31}^1R_{14}^4+\frac{1}{2}R_{24}^4+R_{34}^1R_{44}^4+R_{34}^2=0,\\
R_{31}^1R_{24}^4+-\frac{1}{2}R_{14}^4+R_{34}^2R_{44}^4-R_{34}^1=0,\\
R_{31}^1=R_{32}^2,\\
R_{34}^{4}=s\in\mathbb{R}.
\end{array}
\right.\eqno(7)
\end{equation*}

特别的当$R_{14}^4\neq 0,\ R_{24}^4=0$,可以得到$R_{41}^1=R_{42}^1=R_{44}^1=R_{42}^2=R_{44}^2=0 $.

综上所述,上述方程组刻画了Oscillator李代数上的一类左对称代数结构,由受限于方程(7)的六元组$(R_{31}^1,R_{34}^1,R_{34}^2,R_{14}^4,R_{24}^4,R_{44}^4)$刻画.
\end{frame}

\begin{frame}[plain]
	\frametitle{本文的主要结果}
	\begin{Example}
		取$R_{31}^1=R_{44}^4=2,R_{34}^1=R_{34}^2=R_{14}^4=R_{24}^4=0$,则由上述结果,可令Oscillator李代数上的一类左对称代数,其满足以下方程组
		
		\begin{equation*}\left\{
		\begin{array}{l}
		[e_1,e_2]=e_3,[e_4,e_1]=e_2,[e_4,e_2]=-e_1,\\
		e_1e_1=2e_3,e_1e_2=\frac{1}{2}e_3,\\
		e_2e_1=-\frac{1}{2}e_3,e_2e_2=2e_3,\\
		e_4e_1=e_2,e_4e_2=-e_1,e_4e_4=se_3+2e_4.
		\end{array}
		\right.
		\end{equation*}
		
		$$L_1=\left({\begin{smallmatrix}
			 0 & 0 & 0 & 0 \\
			 0 & 0 & 0 & 0  \\
			 2 & \frac{1}{2} & 0 & 0  \\
			 0 & 0 & 0 & 0
			\end{smallmatrix}} \right),\quad L_2=\left({\begin{smallmatrix}
			 0 & 0 & 0 & 0 \\
			 0 & 0 & 0 & 0  \\
			 -\frac{1}{2} & 2 & 0 & 0  \\
			 0 & 0 & 0 & 0
			\end{smallmatrix}} \right),\quad L_3=0,\quad L_4=\left({\begin{smallmatrix}
			 0 & -1 & 0 & 0 \\
			 1 & 0 & 0 & 0  \\
			 0 & 0 & 0 & s  \\
			 0 & 0 & 0 & 2
			\end{smallmatrix}} \right).$$
		
		$$R_1=\left({\begin{smallmatrix}
			 0 & 0 & 0 & 0 \\
			 0 & 0 & 0 & 1  \\
			 2 & -\frac{1}{2} & 0 & 0  \\
			 0 & 0 & 0 & 0
			\end{smallmatrix}} \right),\quad R_2=\left({\begin{smallmatrix}
			 0 & 0 & 0 & -1 \\
			 0 & 0 & 0 & 0  \\
			 \frac{1}{2} & 2 & 0 & 0  \\
			 0 & 0 & 0 & 0
			\end{smallmatrix}} \right),\quad R_3=0,\quad R_4=\left({\begin{smallmatrix}
			 0 & 0 & 0 & 0 \\
			 0 & 0 & 0 & 0  \\
			 0 & 0 & 0 & s  \\
			 0 & 0 & 0 & 2
			\end{smallmatrix}} \right).$$
		因为$tr(R_4)=2$,所以此类左对称代数是Oscillator李代数上的一类非完备左对称代数.
		\end{Example}
\end{frame}

\begin{frame}
	\zihao{-2}\centering{\textbf{谢谢各位专家的聆听!}}
\end{frame}
\end{document}